\documentclass[cyan]{elegantnote}

\author{陈传升}
\email{sheng\_ccs@163.com}
\zhtitle{读《矢算场论札记》}	
\entitle{Reading}
\version{1.00}
\myquote{Victory won\rq t come to us unless we go to it.}
\logo{logo.pdf}
\cover{cover.pdf}

%green color
   \definecolor{main1}{RGB}{210,168,75}
   \definecolor{seco1}{RGB}{9,80,3}
   \definecolor{thid1}{RGB}{0,175,152}
%cyan color
   \definecolor{main2}{RGB}{239,126,30}
   \definecolor{seco2}{RGB}{0,175,152}
   \definecolor{thid2}{RGB}{236,74,53}
%cyan color
   \definecolor{main3}{RGB}{127,191,51}
   \definecolor{seco3}{RGB}{0,145,215}
   \definecolor{thid3}{RGB}{180,27,131}

\usepackage{makecell}
\usepackage{lipsum}

\begin{document}
	\maketitle
	\tableofcontents

%	更改了cls文件。\newdef 改为Property ,因此在使用性质的框框的时候,用这个即可
%

%		\chapter*{写在前面}
	
	一直以来都想要好好的完善一下自己的数理知识,同时也愉快的使用一次\LaTeX,经过了之前写matlab的使用熟悉了GitHub。终于下定决心\LaTeX 和GitHub结合一下,用这种方式记录下自己的第一个电子版的读书笔记。(嗯。其实手写的读书笔记也没有)
	
	\LaTeX 的模板取自于Elegant Note模板,得到的作者的唯一联系方式是他的邮箱 \url{ddswhu@gmail.com }特此感谢。
	
	《矢算场论札记》梁昌洪著,书和“大佬”借的。选择这本书作为自己的一个开始,一个原因是学科需要,另一也是对梁老师有特殊的好感。有好感的原因呢,一是因为我女朋友也在西电,另一个则是因为梁昌洪老师的《简明微波》一书。
	
	2018年12月09日下载模板,2018年12月10日,正式开始这个笔记的记录,不知道多年之后的自己看见了,会是什么感觉。
	\addcontentsline{toc}{chapter}{写在前面}
	
%	\chapter{矢量}

凡是与三个独立因素有关的物理量均可以采用三维矢量表示。由此物理量可以分为标量,矢量,二阶张量等。在三维空间中,一个二阶张量则有9个分量,可以表示为一个有序9元数组或3×3阶的矩阵


\section{矢量与矩阵}

矢量
\[\vec{A}=A_x\widehat{i}+A_y\widehat{j}+A_z\widehat{k}\]


\noindent	对应矩阵
\[A=\left[ \begin{array}{c}
A_x\\
A_y\\
A_z\\
\end{array} \right] \]



\section{矢量混合积}
表征平行六面体有向体积
\[
\vec{A}\cdot \left( \vec{B}\times \vec{C} \right) =\vec{B}\cdot \left( \vec{C}\times \vec{A} \right) =\vec{C}
\]

\begin{newdef}[]
	三个非零矢量混合积为0的充要条件是$\vec{A},\vec{B},\vec{C}$三个矢量共面,对应有向体积为0。
\end{newdef}




\section{矢量二重叉积}	
\[
\vec{A}\times \left( \vec{B}\times \vec{C} \right) =\vec{B}\left( \vec{A}\cdot \vec{C} \right) -\vec{C}\left( \vec{A}\cdot \vec{B} \right) 
\]
\begin{newdef}[]
	三个非零矢量混合积为0的充要条件是$\vec{A},\vec{B},\vec{C}$三个矢量共面,对应有向体积为0。
\end{newdef}


\section{Laplace公式}
\[
\left( \vec{A}\times \vec{B} \right) \cdot \left( \vec{C}\times \vec{D} \right) =\left( \vec{A}\cdot \vec{C} \right) \left( \vec{B}\cdot \vec{D} \right) -\left( \vec{A}\cdot \vec{D} \right) \left( \vec{B}\cdot \vec{C} \right) 
\]


\section{Lagrange公式}
\[
\left| \vec{A}\times \vec{B} \right|^2+\left| \vec{A}\cdot \vec{B} \right|^2=\left| \vec{A} \right|^2\left| \vec{B} \right|^2
\]


\section{矢量的除法}
 矢量的叉乘无法唯一性定义矢量的除法运算。只有利用“$\cdot$”和“$\times$”同时定义才可以。如下,已知$\vec{a}$,$\vec{d}$和标量$c$,求解$\vec{b}$

\begin{equation*}
\left\{ \begin{array}{c}
\vec{a}\cdot \vec{b}=c\\
\vec{a}\times \vec{b}=\vec{d}\\
\end{array} \right. 
\end{equation*}
构造
\[
\vec{a}\times \left( \vec{a}\times \vec{b} \right) =\vec{a}\left( \vec{a}\cdot \vec{b} \right) -\vec{b}\left( \vec{a}\cdot \vec{a} \right) 
\]
易得
\[
\vec{b}=\frac{\vec{a}\left( \vec{a}\cdot \vec{b} \right) -\vec{a}\times \left( \vec{a}\times \vec{b} \right)}{\left( \vec{a}\cdot \vec{a} \right)}=\frac{\vec{a}c-\vec{a}\times \vec{d}}{\left( \vec{a}\cdot \vec{a} \right)}
\]


从矢量除法的定义不难看出,矢量的“$\cdot$”和“$\times$”是相互关联且互为补充的。
这里退化到平面矢量( 可作复数的对应),做出讨论。有
\[
\left\{ \begin{array}{c}
\vec{a}=a_x\hat{i}+a_y\hat{j}=a_x+ia_y\\
\vec{b}=b_x\hat{i}+b_y\hat{j}=b_x+ib_y\\
\end{array} \right. 
\]
于是有
\[
\bar{a}b=\left( \vec{a}\cdot \vec{b} \right) +\mathit{i}\left\{ \vec{a}\times \vec{b} \right\} 
\]
其中$\bar{a}$表示$a$的共轭复数;$\left\{\right\}$符号表示不计算方向,只计算正负的叉积运算。

\noindent 易得
\[
\vec{b}=\frac{\vec{a}\left( \vec{a}\cdot \vec{b} \right) +\mathit{i}a\left( \vec{a}\times \vec{b} \right)}{\left| \mathit{a} \right|^2}
\]
书中后面讨论的Hamilton算子$\nabla $、散度$\nabla\cdot A$和旋度$\nabla\times\vec{A}$也正好是一对运算。

	\chapter{矢量及矢量分析}
对应书中的第一二章节
%	\begin{appendix}  
	\chapter{常见矢量公式}
	矢量混合积
	\[
	\vec{A}\cdot \left( \vec{B}\times \vec{C} \right) =\vec{B}\cdot \left( \vec{C}\times \vec{A} \right) =\vec{C}
	\]
	
	
	矢量二重叉积
	\[
	\vec{A}\times \left( \vec{B}\times \vec{C} \right) =\vec{B}\left( \vec{A}\cdot \vec{C} \right) -\vec{C}\left( \vec{A}\cdot \vec{B} \right) 
	\]
	
	
	
		
\end{appendix} 

	
	

	
	
	
	
\end{document}


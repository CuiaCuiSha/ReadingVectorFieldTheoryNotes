\documentclass[cyan]{elegantnote}

\author{陈传升}
\email{sheng\_ccs@163.com}
\zhtitle{读《矢算场论札记》}	
\entitle{Reading}
\version{1.00}
\myquote{Victory won\rq t come to us unless we go to it.}
\logo{logo.pdf}
\cover{cover.pdf}

%green color
   \definecolor{main1}{RGB}{210,168,75}
   \definecolor{seco1}{RGB}{9,80,3}
   \definecolor{thid1}{RGB}{0,175,152}
%cyan color
   \definecolor{main2}{RGB}{239,126,30}
   \definecolor{seco2}{RGB}{0,175,152}
   \definecolor{thid2}{RGB}{236,74,53}
%cyan color
   \definecolor{main3}{RGB}{127,191,51}
   \definecolor{seco3}{RGB}{0,145,215}
   \definecolor{thid3}{RGB}{180,27,131}

\usepackage{makecell}
\usepackage{lipsum}

\begin{document}
	\maketitle
	\tableofcontents
	\chapter{写在前面}
	一直以来都想要好好的完善一下自己的数理知识,同时也愉快的使用一次\LaTeX,经过了之前写matlab的使用熟悉了GitHub。终于下定决心\LaTeX 和GitHub结合一下,用这种方式记录下自己的第一个电子版的读书笔记。(嗯。其实手写的读书笔记也没有)
	
	\LaTeX 的模板取自于Elegant Note模板,得到的作者的唯一联系方式是他的邮箱 \url{ddswhu@gmail.com }特此感谢。
	
	《矢算场论札记》梁昌洪著,书是和“大佬”借的。选择这本书作为自己的一个开始,一个原因是学科需要,另一也是对梁老师有特殊的好感。有好感的原因呢,一是因为我女朋友也在西电,另一个则是因为梁昌洪老师的《简明微波》一书。
	
	2018年12月09日下载模板,2018年12月10日,正式开始这个笔记的记录,不知道多年之后的自己看见了,会是什么感觉。
	
	
	\chapter{矢量及矢量分析}
对应书中的第一二章节
	
	
	
\end{document}


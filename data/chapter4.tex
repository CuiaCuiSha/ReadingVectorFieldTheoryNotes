\chapter{梯度}
如果说\textbf{等值面}是数量场的宏观特征,那么场的空间变化就是其\textbf{微观特征}。
\section{$ \hat{l} $的 方向余弦}
数量场$ u=u\left(x,y,z\right) $的变化空间取一点$ M_0 $,对应$ u_0=u\left(x_0,y_0,z_0\right) $,研究此时场朝$ \hat{l} $方向的变化规律。

任取一段$ \Delta \hat{l} $,写出
\[ \Delta \hat{l}=\Delta x\hat{i}+\Delta y\hat{j}+\Delta z\hat{k} \]
式中$ \Delta x $,$ \Delta y $,$ \Delta z $表示$ \Delta \hat{l} $在$ x$,$y$,$z $轴的投影,则可知这一方向的单位矢是
\[\hat{l}=\left( \frac{\Delta x}{\Delta l} \right) \hat{i} +\left( \frac{\Delta y}{\Delta l} \right) \hat{j}+\left( \frac{\Delta z}{\Delta l} \right) \hat{k} \]
则矢量$ \Delta \hat{l} $的长度(或模),具体有,
\[ \Delta l=\sqrt{\left( \Delta x \right) ^2+\left( \Delta y \right) ^2+\left( \Delta z \right) ^2} \]
定义
\[
\cos \alpha =\frac{\Delta x}{\Delta l}\text{,}\cos \beta =\frac{\Delta y}{\Delta l},\cos \gamma =\frac{\Delta z}{\Delta l}
\]
分别表示单位矢$ \hat{l} $在$ x$,$y$,$z $轴的方向余弦,最后得到
\[
\hat{l}=\cos \alpha \hat{i}+\cos \beta \hat{j}+\cos \gamma \hat{k}
\]
且满足等式
\[
\cos ^2\alpha +\cos ^2\beta +\cos ^2\gamma =1
\]

\section{方向导数$ \frac{\partial u}{\partial l} $}
数量场$ u=u\left(x,y,z\right) $在$ M_0\left(x_0,y_0,z_0\right) $处可微,则$ u $在$ M_0 $处沿$ \hat{l} $的方向导数必定存在,且有
\[
\left. \frac{\partial u}{\partial l} \right|_{M_0}=\frac{\partial u}{\partial x}\cos \alpha +\frac{\partial u}{\partial y}\cos \beta +\frac{\partial u}{\partial z}\cos \gamma 
\]

\section{梯度}
重新观察上式,且将他写成两个矢量函数的\textbf{点积形式},即
\begin{equation*}
\begin{aligned}
\left. \frac{\partial u}{\partial l} \right|_{M_0}&=\frac{\partial u}{\partial x}\cos \alpha +\frac{\partial u}{\partial y}\cos \beta +\frac{\partial u}{\partial z}\cos \gamma 
\\
&=\left( \frac{\partial u}{\partial x}\hat{i}+\frac{\partial u}{\partial y}\hat{j}+\frac{\partial u}{\partial z}\hat{k} \right) \cdot \left( \cos \alpha \hat{i}+\cos \beta \hat{j}+\cos \gamma \hat{k} \right) 
\\
&=\left( \frac{\partial u}{\partial x}\hat{i}+\frac{\partial u}{\partial y}\hat{j}+\frac{\partial u}{\partial z}\hat{k} \right) \cdot \hat{l}
\end{aligned}
\end{equation*}
上式清楚的表明:数量场的方向导数由两部分组成:一个是方向单位矢,一个是数量场的导数矢$ \frac{\partial u}{\partial x}\hat{i}+\frac{\partial u}{\partial y}\hat{j}+\frac{\partial u}{\partial z}\hat{k}   $如果考虑\ref{chapter3_3}所引入的Hamilton矢量算子$ \nabla $,
\[
\nabla=\hat{i}\frac{\partial}{\partial x}
+\hat{j}\frac{\partial}{\partial y}
+\hat{k}\frac{\partial}{\partial z}
\]
即可写出
\[
\nabla u=  \left(  \hat{i}\frac{\partial}{\partial x}
+\hat{j}\frac{\partial}{\partial y}
+\hat{k}\frac{\partial}{\partial z}
 \right) u 
 \]
 于是有
\[
\frac{\partial u}{\partial l}=\left( \nabla u \right) \cdot \hat{l}
\]

定义数量场$ u\left(x,y,z\right) $在点$ M_0 $处的梯度为
\[
\nabla u=\frac{\partial u}{\partial x}\hat{i}+\frac{\partial u}{\partial y}\hat{j}+\frac{\partial u}{\partial z}\hat{k}
\]

\begin{newdef}[]
	梯度是一个矢量,且是数量场$ u\left(x,y,z\right) $在$ M_0 $处的\textbf{固有属性},与方向$ \hat{l} $无关。

\end{newdef}






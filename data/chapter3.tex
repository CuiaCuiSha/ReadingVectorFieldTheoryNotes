\chapter{场}

场是物理量的\textbf{空间函数},\textbf{时间函数}。根据物理量的性质,场可以分为数量场,矢量场,张量场。


\begin{newdef}[]
	场有两个显著的特点:
		
		场是物理的客观存在,不以坐标系选取而变化。
		
		场随时间空间联合变化,本章讨论的是不随时间变化的稳定场论。
	
\end{newdef}

\section{数量场}
数量函数\textit{u}是点\textit{M}的函数$u=u(m)$。进一步写成\[u=u(x,y,z)\]

重要的宏观特征:\textbf{等值面}。

\section{矢量场}
如果空间中任意点\textit{M}对应一矢量函数
\[\vec{A}=\vec{A}\left(M\right)  \] 
则称$\vec{A}$是此空间的一个矢量场,对于一个直角坐标系,有
\[\vec{A}=\vec{A}\left(x,y,z\right)\]

重要的宏观特征:\textbf{矢量线},简称矢线。例如Faraday磁力线。


\section{Hamilton算子}\label{chapter3_3}
场的空间变化确定其细微的特征。它的研究方法是考虑场的空间微分和导数,并给出相应分析。为此在三维情况引入
\[
\nabla=\hat{i}\frac{\partial}{\partial x}
		+\hat{j}\frac{\partial}{\partial y}
		+\hat{k}\frac{\partial}{\partial z}
\]
表示场与空间相互作用的Hamilton矢量算子。

\begin{newdef}[]
$ \nabla $表示一个运算符号,本身缺乏独立的意义。和场结合才有作用。

$ \nabla $算子具有矢量和运算的双重特性,这是认识算子的一个极为重要的概念。
\end{newdef}

$q_1$,$q_2$和$q_3$表示独立的正交坐标,$\hat{e}_1$,$\hat{e}_2$和$ \hat{e}_3 $则表示其相应的单位切向矢。广义正交曲线坐标的$ \nabla $算子的一般形式为,

\[
\nabla =\hat{e}_1\frac{1}{H_1}\frac{\partial}{\partial q_1}+\hat{e}_2\frac{1}{H_2}\frac{\partial}{\partial q_2}+\hat{e}_3\frac{1}{H_3}\frac{\partial}{\partial q_3}
\]
式中,$ H_1 $,$ H_2 $和$ H_3 $为Lam$\grave{\text{e}}$ 系数。

%\begin{table}[htbp]
%	
%	\centering
%	\caption{矢量函数的基本性质}		 
%	\begin{tabular}{c|c|c|c}
%		\toprule	{.7}
%			
%%		\midrule	{.7/textwidth}
%		
%		\bottomrule	{0.7}
%	\end{tabular}
%\end{table}


\begin{table}
	\centering
	\caption{description}
	\begin{spacing}{2.5}		
	\begin{tabular}{cccc}
		\toprule  %添加表格头部粗线
		
		
		坐标系& $ q_1$\ $ q_2$ \  $ q_3 $& $ H_1$\ $ H_2$\ $ H_3$& $ \nabla $\\
		\midrule  
		直角坐标系& \textit{x y z}& 1 1 1 &2\\
		圆柱坐标系& $\rho$\ $ \phi$\ $ z$& 1\ $ \rho $\ 1&1\\
		球坐标系  & $r$\ $\theta$\ $\phi$&1\ $r$\  $r\sin\theta$&17\\
		
		
		
		\bottomrule %添加表格底部粗线
	\end{tabular}
	\end{spacing}
\end{table}

\section{坐标单位矢}
同一矢量场在不同的坐标系下有不同的表象和形式。

\begin{newdef}[]
必须强调指出,三种坐标系中,只有直角坐标系的$ \hat{i} $,$ \hat{j} $,$ \hat{k} $和圆柱坐标系的$ \hat{e}_z $是不变单位矢,其他的各种单位矢,例如$ \hat{e}_\rho $
,$ \hat{e}_\phi $;$ \hat{e}_r $,$ \hat{e}_\phi $,$ \hat{e}_\theta $均为\textbf{变单位矢}。换句话说,方向是时刻在变化的。\textbf{因此必须参与微分和积分}。
\end{newdef}







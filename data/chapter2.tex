\chapter{矢量分析}
\section{标量函数和矢量函数}
对于标量函数,标量\textit{u}随着参量\textit{t}的变化而变化,即
\[u=u(t)\]
\noindent
对于矢量$\vec{A}$随参数t变化,即
\[
\vec{A}=\vec{A}\left(t\right)
\]
则称$\vec{A}$为矢量函数。分量形式的表示为
\[
\vec{A}\left(t\right)=A_x\left( t \right) \hat{i}+A_y\left( t \right) \hat{j}+A_z\left( t \right) \hat{k}
\]


\begin{newdef}[]
	一个矢量函数$\vec{A}\left(t\right)$实际上是由三个\textbf{独立}有序的数量函数$A_x\left(t\right)$,$A_y\left(t\right)$和$A_y\left(t\right)$结合而成的。
	
	这一点和复解析函数不同,复解析函数$w=u+\textit{i}v$也是由两个二元实函数结合而成,但\textit{u,v}之间并不独立,收到Cauchy-Riemann条件约束。
	
	关于Cauchy-Riemann条件,可以参考相关的复变函数书籍,或者是梁昆淼所著的《数学物理方法》一书。
\end{newdef}

\section{矢量函数的导数和微分}

\begin{newdef}[]
	矢量函数的导数是一个矢量,它是矢端曲线的切线,并\textbf{始终}指向对应\textit{t}增大的方向。
\end{newdef}



\section{矢量导数的应用}
	
	
\section{矢量函数的积分}
对于一个矢量表示成下面的形式
	\[
	\vec{A}\left(t\right)=A_x\left( t \right) \hat{i}+A_y\left( t \right) \hat{j}+A_z\left( t \right) \hat{k}
	\]
	则矢量函数的不定积分有
	\[
	\int{\vec{A}\left( t \right)}\,\,\mathit{d}t=\hat{i}\int{A_x\left( t \right)}\,\,\mathit{d}t+\hat{j}\int{A_y\left( t \right)}\,\,\mathit{d}t+\hat{k}\int{A_z\left( t \right)}\,\,\mathit{d}t
	\]
\begin{newdef}[]
	矢量函数的积分实质上是三个\textbf{独立有序}的数量函数的不定积分。
\end{newdef}

	

\begin{table}[htbp]
	
	\centering
	\caption{矢量函数的基本性质}		 
	\begin{tabular}{c}
		\toprule	
		$
		\int{k\vec{A}\left( t \right)}\,\,\mathit{d}t=k\int{\vec{A}\left( t \right)}\,\,\mathit{d}t
		$\\
		$
		\int{\left[ \vec{A}\left( t \right) \pm \vec{B}\left( t \right) \right]}\,\,\mathit{d}t=\int{\vec{A}\left( t \right)}\,\,\mathit{d}t\pm \int{\vec{B}\left( t \right)}\,\,\mathit{d}t
		$\\
		$
		\int{\vec{a}u\left( t \right)}\,\,\mathit{d}t=\vec{a}\int{u\left( t \right)}\,\,\mathit{d}t
		$\\
		$
		\int{\vec{a}\cdot \vec{A}\left( t \right)}\,\,dt=\vec{a}\cdot \int{\vec{A}\left( t \right)}\,\,dt
		$\\
		$
		\int{\vec{a}\times \vec{A}\left( t \right)}\,\,dt=\vec{a}\times \int{\vec{A}\left( t \right)}\,\,dt
		$\\
		
		\bottomrule	
	\end{tabular}
\end{table}

	
	

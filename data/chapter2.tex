\chapter{矢量分析}
\section{标量函数和矢量函数}
对于标量函数,标量\textit{u}随着参量\textit{t}的变化而变化,即
\[u=u(t)\]
\noindent
对于矢量$\vec{A}$随参数t变化,即
\[
\vec{A}=\vec{A}\left(t\right)
\]
则称$\vec{A}$为矢量函数。分量形式的表示为
\[
\vec{A}\left(t\right)=A_x\left( t \right) \hat{i}+A_y\left( t \right) \hat{j}+A_z\left( t \right) \hat{k}
\]


\begin{newdef}[]
	一个矢量函数$\vec{A}\left(t\right)$实际上是由三个\textbf{独立}有序的数量函数$A_x\left(t\right)$,$A_y\left(t\right)$和$A_y\left(t\right)$结合而成的。
	
	这一点和复解析函数不同,复解析函数$w=u+\textit{i}v$也是由两个二元实函数结合而成,但\textit{u,v}之间并不独立,收到Cauchy-Riemann条件约束。
\end{newdef}

\section{矢量函数的导数和微分}